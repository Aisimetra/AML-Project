\section{Introduzione}
%\textit{The introduction should provide a clear statement of the problem posed by the project, and why the problem is of interest. It should reflect the scenario, if available. If needed, the introduction also needs to present background information so that the reader can understand the significance of the problem. A brief summary of the hypotheses and the approach your group used to solve the problem should be given, possibly also including a concise introduction to theory or concepts used later to analyze and to discuss the results.}


L'analisi condotta è volta a costruire una rete neurale che possa identificare quali persone, tra quasi 10'000 persone costaricane esaminate, si trovino in uno stato di povertà ed abbiano bisogno di assistenza. Il dataset utilizzato riporta per ogni persona una serie di informazioni personali e familiari.
A un'estesa analisi delle variabili, volta a capire quante di queste fossero effettivamente utili e significative, eventualmente aggregandole, sono seguite due diverse tecniche di preprocessing dei dati (FAMD e OHE) volte ad ottenere rappresentazioni utili a rendere più efficace l'addestramento della rete neurale. 
Scelto il preprocessing migliore, sono quindi state sperimentate diverse tipologie di reti e di regolarizzazioni per attenuare l'overfitting: \textit{L1}, \textit{L2}, \textit{Dropout}, \textit{Early Stopping} ed una tecnica per gestire lo sbilanciamento della classe target chiamata \textit{Class Weight}.